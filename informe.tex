\documentclass{article}
\usepackage{amsmath}
\usepackage[a4paper, left=2.5cm, right=2.5cm, top=2.5cm, bottom=2.5cm]{geometry}  % Paquete para márgenes


\begin{document}

\section*{Ejercicio 1: Modelo básico}

\[
\parbox{\textwidth}{En primer lugar, con el objetivo de minimizar los costos de distribución relacionados con la apertura de centrales para abastecer a las oficinas y las distancias de conexión, se planteó el siguiente modelo básico.}
\]

\subsection*{Conjuntos:}
\[
C = \left\{ 1, 2, \ldots, 10 \right\} = \text{Conjunto de centrales operativas extraído de "centrales.txt"}
\]
\[
O =\left\{ 1, 2, \ldots, 56 \right\} = \text{Conjunto de oficinas extraído de "oficinas.txt"}
\]

\vspace{0.5cm}


\subsection*{Variables:}
\[
x_{ij} = 
\begin{cases} 
1 & \text{si la oficina } i \text{ está conectada a la central } j \\
0 & \text{en otro caso}
\end{cases}
\]

\vspace{0.5cm}

\[
y_j = 
\begin{cases} 
1 & \text{si la central } j \text{ está abierta} \\
0 & \text{en otro caso}
\end{cases}
\]

\subsection*{Función objetivo:}

Minimizar:
\[
 \sum_{j \in C} \text{costo\_central} \cdot y_j + \sum_{i \in O}\sum_{j \in C} \text{d}_{ij} \cdot \text{costo\_cable} \cdot x_{ij}
\]

\[
\parbox{\textwidth}{Siendo que el costo de abrir una central de operaciones es de 5700 y que el costo de 1 metro de cable es de 17/1000, la función objetivo en efecto es la siguiente:}
\]

\[
 \sum_{j \in C} \text{5700} \cdot y_j + \sum_{i \in O}\sum_{j \in C} \text{d}_{ij} \cdot \text{17/1000} \cdot x_{ij}
\]

\subsection*{Restricciones:}

\begin{itemize}
    \item \text{Cada oficina debe estar conectada a una única central operativa:}
    \[
    \sum_{j \in C} x_{ij} = 1     \hspace{1cm} \forall i \in O
    \]
    
    \vspace{0.3cm}
    
    \item \text{Si una oficina está asignada a una central, esa central debe estar abierta:}
    \[
    x_{ij} \leq y_j  \hspace{1cm}\forall i \in O, \forall j \in C
    \]
    
    \vspace{0.3cm}

    \item\parbox{\textwidth}{La suma de las demandas de las oficinas conectadas a una central no debe exceder la capacidad máxima (\(M = 15000\)):}
    \[
   \sum_{i \in O} x_{ij} \cdot \text{demanda}_i \leq M \hspace{1cm} \forall j \in C
    \]
\end{itemize}

\section*{Ejercicio 2: Restricción adicional}
\[
\parbox{\textwidth}{Se puede agregar una restricción adicional al modelo para que en la solución se considere que una central no pueda atender a más de 10 oficinas, como una manera de hacer las distribuciones más equitativas y no permitir la sobrecarga de una central particular más allá de que pueda cumplir con las restricciones de demanda.}
\]


\begin{itemize}
    \item \text{Cada central no puede atender a más de 10 oficinas:}
    \[
   \sum_{i \in O} x_{ij} \leq 10 \hspace{1cm}\forall j \in C
    \]
\end{itemize}

\section*{Ejercicio 3: Parámetro adicional}

\[
\text{capacidad\_mínima} = \frac{\sum_{i \in O} \text{demanda}_i}{10}
\]

\end{document}
