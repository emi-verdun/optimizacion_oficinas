\documentclass{article}
\usepackage{amsmath}
\usepackage[a4paper, left=2.5cm, right=2.5cm, top=2.5cm, bottom=2.5cm]{geometry}  % Paquete para márgenes

\begin{document}

\section*{Ejercicio 1: Modelo básico}

En primer lugar, con el objetivo de minimizar los costos de distribución relacionados con la apertura de centrales para abastecer a las oficinas y las distancias de conexión, se planteó el siguiente modelo básico.

\subsection*{Conjuntos:}
\[
C = \left\{ 1, 2, \ldots, 10 \right\} = \text{Conjunto de centrales operativas extraído de "centrales.txt"}
\]
\[
O = \left\{ 1, 2, \ldots, 56 \right\} = \text{Conjunto de oficinas extraído de "oficinas.txt"}
\]

\vspace{0.5cm}

\subsection*{Variables:}
\[
x_{ij} = 
\begin{cases} 
1 & \text{si la oficina } i \text{ está conectada a la central } j \\ 
0 & \text{en otro caso}
\end{cases}
\]

\vspace{0.5cm}

\[
y_j = 
\begin{cases} 
1 & \text{si la central } j \text{ está abierta} \\ 
0 & \text{en otro caso}
\end{cases}
\]

\subsection*{Función objetivo:}

Minimizar:
\[
\sum_{j \in C} \text{costo\_apertura} \cdot y_j + \sum_{i \in O} \sum_{j \in C} \text{d}_{ij} \cdot \text{costo\_cable} \cdot x_{ij}
\]

Siendo que el costo de abrir una central de operaciones es de 5700 y que el costo de 1 metro de cable es de \(17/1000\), la función objetivo es:

\[
\sum_{j \in C} 5700 \cdot y_j + \sum_{i \in O} \sum_{j \in C} \text{d}_{ij} \cdot \frac{17}{1000} \cdot x_{ij}
\]

\subsection*{Restricciones:}

\begin{itemize}
    \item Cada oficina debe estar conectada a una única central operativa:
    \[
    \sum_{j \in C} x_{ij} = 1 \quad \forall i \in O
    \]
    
    \item Si una oficina está asignada a una central, esa central debe estar abierta:
    \[
    x_{ij} \leq y_j \quad \forall i \in O, \forall j \in C
    \]

    \item La suma de las demandas de las oficinas conectadas a una central no debe exceder la capacidad máxima (\(M = 15000\)):
    \[
    \sum_{i \in O} x_{ij} \cdot \text{operaciones}_i \leq M \quad \forall j \in C
    \]
\end{itemize}

\subsection*{Resultados:}

Con el modelo básico planteado, se encontró un costo mínimo de 34549.843. Dicho mínimo se alcanzó en 9 segundos por el solver y asigna las oficinas a las centrales de la siguiente manera:
\vspace{0.5cm}

\begin{itemize}
    \item\text{Centrales abiertas:} 1, 2, 3, 5, 7, 9.
	\begin{itemize}
	\item Oficinas asignadas a la central 1: 15, 46, 47, 48, 49, 50, 51, 52, 53, 54
	\item Oficinas asignadas a la central 2: 1, 2, 3, 14, 16, 29, 30, 42, 43, 44
	\item Oficinas asignadas a la central 3: 10, 11, 18, 19, 21, 31, 32, 33
	\item Oficinas asignadas a la central 5: 8, 9, 17, 20, 34, 35, 36, 37, 40, 41
	\item Oficinas asignadas a la central 7: 4, 5, 6, 7, 26, 28, 38, 45, 55, 56
	\item Oficinas asignadas a la central 9: 12, 13, 22, 23, 24, 25, 27, 39
	\end{itemize}
\end{itemize}



\section*{Ejercicio 2: Restricción adicional}
\[
\parbox{\textwidth}{Se puede agregar una restricción adicional al modelo para que en la solución se considere que una central no pueda atender a más de 10 oficinas, como una manera de hacer las distribuciones más equitativas y no permitir la sobrecarga de una central particular más allá de que pueda cumplir con las restricciones de demanda.}
\]


\begin{itemize}
    \item \text{Cada central no puede atender a más de 10 oficinas:}
    \[
   \sum_{i \in O} x_{ij} \leq 10 \hspace{1cm}\forall j \in C
    \]
\end{itemize}
\subsection*{Resultados con restricción adicional:}
\[
\parbox{\textwidth}{Teniendo en cuenta esta nueva restricción, el óptimo encontrado no cambia. Fijándonos en el óptimo con el modelo sin esta restricción adicional, a ninguna central se le asignan más de 10 oficinas, por lo que esta restricción no afecta al óptimo. Si se pusiera una menor cantidad de oficinas máximas a las que puede abastecer una central, ya nos encontraríamos con otro caso distinto.}
\]
\[
\parbox{\textwidth}{Igualmente, agregando esta restricción se nota una pequeña mejora en el tiempo de ejecución del solver, pasando de tardar 8 segundos en encontrar la solución óptima a tardar 7 segundos. Si bien no parece muy significante, con mayores instancias o cambios en los datasets, esta restricción podría ser de ayuda para la reducción de tiempo de ejecución.}
\]

\section*{Ejercicio 3: Parámetro adicional}

\[
\text{capacidad\_mínima} = \frac{\sum_{i \in O} \text{demanda}_i}{10}
\]

\end{document}
